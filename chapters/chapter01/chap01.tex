%
% File: chap01.tex
% Author: Abhushan Sahu
% Description: Introduction chapter where the stuff goes on.
%

\let\textcircled=\pgftextcircled

\chapter{Introduction}
\label{chap:intro}

\initial{W}e live in a terrible little world, chaos and order hang in balance. People tend to think as there are no strings attached to them, but reality is a bitter truth. \\
In the words of Ultron  ``\emph{ I'm gonna show you something beautiful. You want to protect the world, but you don't want it to change. You are all puppets, tangled in strings,strings. }''
With that being said, it is time to adapt or evolve, and the former one just made it through the narrows of bits and bytes to world of scripts and christ. \\



%=======
\section{Objective	/ Purpose}
\label{sec:sec01}

Imagine a world, where the care could reach the needed in time, thereby affecting the living to mortality ratio in a positive sign.
Chaos and order, as listed above are just the way of saying life and nature; unpredictable. While making this application, we were clear to ourselves that nature can't be tamed, what can be controlled is its aftermath. 
It is more of a follow-up system than a preventive one. Thus, it is safe to say that it is a causal system. \\


\section{Problem Definition}
\label{sec:sec02}

The main system that stretches us out of the comfort was the matter of human safety.
We had seen many development in categories such as women safety, health-care, and others. But what was critical was still missing out, leaving the needed to bite the dust. \\
Our main concern was not bigoted, but was generalized to prevent the life-threatening injuries to actually threaten the lives.\\
(...)Suppose, a scenario were a human is left hurt, unconscious and alone, that's where this application comes to rescue.

\section{Practicality}
\label{sec:sec03}



The assistance of this algorithm ranges from everyone to everywhere, it is just like a personal insurance in a pocket, but added, it can actually help and save you.\\
Presenting {\Large \textit{` Helpr! ' }} \\
\begin{figure}[h]
	\centering
	\includegraphics[height=0.35\textheight]{fig01/icon}
	\mycaption[Launcher icon of the application.]{Launcher icon of the application.}
	\label{fig:RHP01}
\end{figure}
 
 This here, is the implementation of the idea, the buzz of the talk.
A mobile phone application, that tracks the human activity to sense the user's real time behavior.
Thereby ensuring the user's health at all time.\\
\begin{description}
\item[$\cdot$ q1]Why mobile app? Because, in this modern scenario, the only thing that a man never forget to have is his phone.
\item[$\cdot$ q2]Is it effective? Yes, as in terms of sensors, it uses the most rudimentary one, that every phone have. (Had since days)
\item[$\cdot$ q3]Is it cost-effective? Of course, A phone only cost a few thousand bucks, additionally this application has been provided for free.
\item[$\cdot$ q4]Course of action? Detailed study inside.
 \end{description}
 
\section{Technologies Used}
\label{sec:sec04}

For this project to incarnate a mold was needed, and that mold was found in everyone's pocket. \\

\subsection{Software Requirements}
\label{subsec:subsec01}

\begin{description}
\item[$\cdot$ s1] An OS - Android and Windows.
\item[$\cdot$ s2] Java RE.
\item[$\cdot$ s3] Eclipse juno.
\item[$\cdot$ s4] Android 2.1 or later.

\end{description}

\subsection{Hardware  Requirements}
\label{subsec:subsec02}
For this to work, we needed all these listed items, and to our surprise they all come bundled in, a smart-phone**
\begin{description}


\item[$\cdot$ h1] Accelerometer.
\item[$\cdot$ h2] Gps.
\item[$\cdot$ h3] Internet. *
\item[$\cdot$ h4] Gyroscope. *
\item[$\cdot$ h5] Sim.
\item[$\cdot$ h6] Memory.
\item[$\cdot$ h7] Human. *
\item[$\cdot$ h7] A server station. *



*Optional; The more the merrier. \\
**Term used to denote a phone that has the least of listed item and supports parallel processing.
\end{description}



%%A figures matrix.
%\begin{figure}[t!]
%\centering
%\begin{minipage}{3.3cm}
%    \centering
%    \subtop[]{\includegraphics[height=0.28\textheight]{fig01/Nswellings}\label{sf:multiRH02a}}
%\end{minipage}
%\hspace{0.5cm}
%\begin{minipage}{3.3cm}
%    \centering
%    \subtop[]{\includegraphics[height=0.27\textheight]{fig01/Mswellings}\label{sf:multiRH02b}}
%\end{minipage}
%\hspace{1.3cm}
%\begin{minipage}{3.3cm}
%    \centering
%    \subtop[]{\includegraphics[height=0.27\textheight]{fig01/rhd1}\label{sf:multiRH02c}}
%\end{minipage}
%\\ \vspace{0.1cm}
%\begin{minipage}{10cm}
%    \centering
%    \subtop[]{\includegraphics[height=0.145\textheight]{fig01/mutantrhd6}\label{sf:multiRH02d}}
%\end{minipage}
%\\ \vspace{0.1cm}
%\begin{minipage}{10cm}
%    \centering
%    \subtop[]{\includegraphics[height=0.16\textheight]{fig01/auxab}\label{sf:multiRH02e}}
%\end{minipage}
%\mycaption[Hair-forming mutant cells.]{(a) A mutant RH cell. Asterisks show multiple sites of RH initiation in a single root hair cell (indicated by the arrows). Figure reproduced from \cite{rigas01}. (b)~Hair-forming cell with three RH initiation locations. The bar represents $50\mu m$. Figure reproduced from \cite{massuci01}. (c) Large bump in mutant {\itshape rhd1}. Figure reproduced from \cite{griersonRH}. (d) Mutant overexpressing gene {\itshape ROP2}; from right-hand to left-hand, numbers indicate progressive snapshots at different times. RH initiation sites are indicated by the arrows. The bar represents $75\mu m$. Figure reproduced from~\cite{mjones01}. (e)~Mutants affected by auxin. On the left-hand side, RH site is farther away from the apical end (left arrow cap); on the right-hand side, multiple RH locations (arrows). Figure reproduced from~\cite{payne01}.}
%\label{fig:multiRH02}
%\end{figure}
%
%% A single figure
%\begin{figure}[t!]
%	\centering
%	\includegraphics[height=0.35\textheight]{fig01/devepzones}
%	\mycaption[Developmental zones of an Arabidopsis root.]{Developmental zones of an Arabidopsis root. Figure reproduced from \cite{griersonRH}.}
%	\label{fig:RHP02}
%\end{figure}
%
%%=========================================================